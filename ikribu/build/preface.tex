\documentclass[10pt]{article}
\usepackage[utf8]{inputenc}
\usepackage[papersize={17in, 11in}]{geometry}
\usepackage[absolute]{textpos}
\TPGrid[0.5in, 0.25in]{23}{24}
\usepackage{nopageno}
\usepackage{palatino}
\usepackage{tabu}
\parindent=0pt
\parskip=12pt
\tabulinesep=1.0mm
\begin{document}

\begin{textblock}{23}(0, 1)
\center \huge PREFACE
\end{textblock}

\begin{textblock}{23}(0, 3)

\textbf{Ikribu} were the songs sung during nightlong vigils held in the cities
of Assyria as early as the 25\textsuperscript{th} century BCE. During the
course of the vigil participants read events of the future in the organs of
animals slit open at the ceremony's start. The liver --- evident font of the
body's blood --- was a particularly rich site of answers. Models of sheeps'
livers excavated from a palace compound in what is now eastern Syria record the
exact locations of the animals' organs to be consulted by magicians. Bodies are
books for reading and the models we have inherited testify to an equation of
marks found on bodies with marks made in clay: at a remove of more than four
thousand years it is now apparent that such an equation informed the actions of
both individuals and the state for centuries longer than even the most enduring
of the reign of kings.

\textbf{Scordatura.} String II of the violin is tuned down a minor third to
F$\sharp$4; string III of the violin is tuned up a major second to E4. String I
of the viola is tuned down a major third to F4; string II of the viola is tuned
up a minor second to E$\flat$4. String IV of the cello is tuned down a major
ninth to B$\flat$0 (an octave below the lowest note on the bass clarinet). The
resulting tunings are these:

\begin{tabu}{l l l}
\phantom{M} & violin: & E5, F$\sharp$4, E4, G3 \\
            & viola: & F4, E$\flat$4, G3, C3 \\
            & cello: & A3, D3, G2, B$\flat$0 \\
\end{tabu}

\textbf{Accidentals.} Accidentals govern only one note. Natural signs are
inserted to clarify the spelling of different pitches following immediately
after each other at the same staff position.

\textbf{String contact points.} Five string contact points appear in the score:

\begin{tabu}{l l l}
\phantom{M} & XT & as close to the fingers as possible (without touching the fingers) \\
            & tasto & very noticeably tasto in color\\
            & pos. ord. & ordinary playing position \\
            & pont. & very noticeably ponticello in color \\
            & XP & as close to the bridge as possible (without touching the bridge) \\
\end{tabu}

\textbf{Bridge contact points.} The indication \textbf{OB} stands for
``directly on the bridge'' and means that the bow should be run diagonally on
the bridge to produce white noise with no pitch at all. Fractional bridge
contact points also appear. These are played with the bow extremely high on the
string such that the hair of the bow runs against both the wrapping of the
string and the wood of the bridge at the same time. Taken as a series these
bridge contact points do three things: they reduce the
fundamental of the string's fingered pitch; they increase the spectral content
of the upper partials; and they replace the overall sensation of pitch with
noise. Some examples:

\begin{tabu}{l l l}
\phantom{M} & XP & as close to the bridge as possible (without touching the bridge) \\
            & $\frac{1}{4}$OB & one quarter of the hair on bridge (and three quarters of the hair on string) \\
            & $\frac{1}{2}$OB & one half of the hair on bridge (and one half of the hair on string) \\
            & $\frac{3}{4}$OB & three quarters of the hair on bridge (and one quarter of the hair on string) \\
            & OB & bow directly on bridge with a diagonal bow (to produce white noise only) \\
\end{tabu}

\textbf{Bow speed colors.} The score contrasts widely different speeds of the bow:
 
\begin{tabu}{l l l}
\phantom{M} & XFB & extremely fast bow (extreme flautando with the bow only very lightly skimming the string) \\
            & FB & fast bow (very pronounced flautando just slightly less than above) \\
            & NBS & normal bow speed (neither flautando nor scratch) \\
            & $\frac{1}{4}$ scratch & timbre with one quarter part scratch (and three quarter parts pitch) \\
            & $\frac{1}{2}$ scratch & timbre with one half part scratch (and one half part pitch) \\
            & $\frac{3}{4}$ scratch & timbre with three quarter parts scratch (and one quarter part pitch) \\
            & scratch moltiss. & timbre with as much scratch (and as little pitch) as possible (though without encouraging subtones) \\
\end{tabu}

Do not substitute tasto for the FB and XFB degrees of bow speed flautando
requested in the score: bow speeds combine freely with the string and bridge
contact points given above. Indications for individuated clicks of the bow also
appear; these result from almost impossibly slow motions of the bow against the
string. \textbf{Glissandi.} Do not rearticulate note-heads in the middle of
glissandi.

\end{textblock}

\begin{textblock}{23}(0, 23)

\textbf{Ikribu} was written for Distractfold who are to premiere the piece on
April 2\textsuperscript{nd} 2016 in Paine Hall on the campus of Harvard
University.

\end{textblock}

\end{document}