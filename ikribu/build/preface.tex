\documentclass[10pt]{article}
\usepackage[utf8]{inputenc}
\usepackage[papersize={17in, 11in}]{geometry}
\usepackage[absolute]{textpos}
\TPGrid[0.5in, 0.25in]{23}{24}
\usepackage{nopageno}
\usepackage{palatino}
\usepackage{tabu}
\parindent=0pt
\parskip=12pt
\tabulinesep=1.0mm
\begin{document}

\begin{textblock}{23}(0, 1)
\center \huge PREFACE
\end{textblock}

\begin{textblock}{23}(0, 3)

\textbf{Ikribu} were the songs sung during nightlong vigils held in the cities
of Assyria. During the course of the vigil participants read the events of the
future in the organs of animals slit open at the ceremony's start. The liver
--- evident font of the body's blood --- constituted a particularly rich site
of answers. Models of sheeps' livers excavated from a palace compound in what
is now eastern Syria record the exact locations of the animals' organs to be
consulted by magicians. From such song-filled inspection of the insides of
animals emerged an understanding of the future vivid enough to inform the
actions of individuals and the decisions of the state.

\textbf{Scordatura.} String II of the violin is tuned down a minor third to
F$\sharp$4; string III of the violin is tuned up a major second to E4. String I
of the viola is tuned down a major third to F4; string II of the viola is tuned
up a minor second to E$\flat$4. String IV of the cello is tuned down a major
ninth to B$\flat$0 (an octave below the lowest note on the bass clarinet):

\begin{tabu}{l l l}
\phantom{M} & violin: & E5, F$\sharp$4, E4, G3 \\
            & viola: & F4, E$\flat$4, G3, C3 \\
            & cello: & A3, D3, G2, B$\flat$0 \\
\end{tabu}

\textbf{Seating positions.} The violinst and violist should sit opposite one
another across an upturned bass drum placed on a chair between them. The
cellist should sit behind the violinist and violist; if a dais is available,
the cellist can be seated atop it to better expose the instrument behind the
bass drum. The clarinetist should be seated at a slight remove from the other
players.

\textbf{Auxilliary instruments.} The players are asked to use the following
auxilliary instruments:

\begin{tabu}{l l l}
\phantom{M} & clarinet: & small stone, slate, large basket of grain \\
            & violin: & small stone, slate, eight cups of grain, bass drum (with brushes) \\
            & viola: & small stone, slate, bass drum (with brushes) \\
            & cello: & small stone, slate, large basket of grain \\
\end{tabu}

\textbf{Stonescratch.} At the end of the piece the cellist is asked to scratch
individuated marks into a piece of slate with a smooth stone. This moment is
marked ``stonescratch'' in the score. Follow the timings and loudnesses given
in the score.

\textbf{Stonecircle.} At different points in the piece the violinst, violist
and clarinetist are asked to move a smooth stone in cirlces across the surface
of a piece of slate. (Flagstone or another stone with a comparable surface can
be used if slate is unavailable.) These moments are marked ``stonecircle'' in
the score. Each of three players should have their own stone and their own
piece of slate; the three pieces of slate should differ slightly from each
other. Circles drawn during performance should be as large as the surface of
the slate will allow and at least as wide in diameter as the width of the
players' palms. Different rates of circling are given as fractions of $\pi$.

\textbf{Grainfall.} Eight times during the piece the violinist is requested to
pour the contents of a cup of barley to the floor. These moments are marked
``grainfall'' in the score. The action is a harbinger of investigation and
should be performed in full view of the audience with grace and deliberation.
The cups should match each other and be big enough to allow grain to be poured
for several seconds continuously as requested by the durations given in the
score. The recepticle used to catch the grain may be made of wood or clay but
not plastic or metal. Another grain may be substituted if barely is
unavailable. Whatever the grain selected, the violinist's choice of grain
should differ from that of the cellist and clarinetist, as described below.
Fill the cups and set them aside before the performance starts.

\textbf{Graincircle.} At different times in the piece the cellist and the
clarinetist are asked to circle their left hands clockwise in a large basket
filled with grain. These moments are marked ``graincircle'' in the score.  The
cellist's and the clarinetist's baskets should be be large enough to hold at
least 5 or 10 kilograms of grain and should be made of wood or clay instead of
plastic or metal. The exact types of grain to use are left to the performers.
But it is important that the cellist and the clarinetist use two different
types of grain and that the cellist's and clarinetist's grains differ from the
violinist's. (Perhaps barley for the violinst, lentils for the cellist and
dried beans for the clarinetist). Different rates of circling are given as
fractions of $\pi$.

\textbf{Amplification.} The cellist's and clarinetist's baskets of grain may be
placed on top of slabs of styrofoam to amplify the the sound of the players'
hands as they move. Electronic amplification should not be used.

\textbf{Accidentals.} Accidentals govern only one note. Natural signs are
inserted to clarify the spelling of different pitches that follow each other at
the same staff position.

\end{textblock}

\begin{textblock}{23}(0, 23)

\textbf{Ikribu} was written for Distractfold who are to premiere the piece on
April 2\textsuperscript{nd} 2016 in Paine Hall on the campus of Harvard
University.

\end{textblock}

\end{document}