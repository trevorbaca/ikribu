\documentclass[10pt]{article}
\usepackage[utf8]{inputenc}
\usepackage[papersize={17in, 11in}]{geometry}
\usepackage[absolute]{textpos}
\TPGrid[0.5in, 0.25in]{23}{24}
\usepackage{nopageno}
\usepackage{tabu}
\usepackage{xltxtra,fontspec,xunicode}
\parindent=0pt
\parskip=12pt
\tabulinesep=1.0mm
\setmainfont{Adobe Garamond Pro}
\begin{document}

\begin{textblock}{23}(0, 1)
\center \huge PREFACE
\end{textblock}

\begin{textblock}{23}(0, 3)

\textbf{Ikribu} were the songs sung in Assyria during nightlong vigils held as
early as the 25\textsuperscript{th} century BCE. During the course of the vigil
participants read events of the future in the organs of animals slit open at
the ceremony's start. Marks visible on the surface of the liver --- font of the
body's blood --- answered questions of importance to the state. Models of
sheep's livers excavated from palace compounds record the exact locations of
the animals' organs to be consulted by magicians. What is surprising is not
that record of the liver-readers and their clients have come down to us. Nor is
it surprising that intimations of the future read on the insides of animals
informed the actions of individuals and decisions of the state. What is
unexpected, at a remove of more than forty centuries, is that information in
the service of statecraft arose to the nighttime accompaniment of song.

\textbf{Seating positions.} The violinst and violist should sit opposite one
another across an upturned bass drum placed on a chair between them. The
cellist should sit behind the violinist and violist; if a dais is available,
the cellist can be seated atop it to better expose the instrument behind the
bass drum. The clarinetist should be seated at a slight remove from the other
players.

\textbf{Scordatura.} String II of the violin is tuned down a minor third to
F$\sharp$4; string III of the violin is tuned up a major second to E4. String I
of the viola is tuned down a major third to F4; string II of the viola is tuned
up a minor second to E$\flat$4. String IV of the cello is tuned down a major
ninth to B$\flat$0 (an octave below the lowest note on the bass clarinet):

\begin{tabu}{l l l}
\phantom{M} & violin: & E5, F$\sharp$4, E4, G3 \\
            & viola: & F4, E$\flat$4, G3, C3 \\
            & cello: & A3, D3, G2, B$\flat$0 \\
\end{tabu}

\textbf{Bow contact point tablature.} Some of the strings' music comprises two
staves. The top staff is notated in a type of bow contact point (BCP) tablature
that specifies which point along the bow is to touch the string at which time.
BCPs are given as fractions between 0 and 1 with 0 indicating the talon and 1
indicating the point; intermedial values indicate a BCP somewhere between the
two. For example, the BCPs $\frac{0}{4} \longrightarrow \frac{1}{4}
\longrightarrow \frac{2}{4}$ indicate that the bow is to be drawn smoothly from
the talon ($\frac{0}{4}$) to a point one quarter of the way from the talon
($\frac{1}{4}$) and finally to exactly the midway point of the bow
($\frac{2}{4}$); the BCPs $\frac{6}{7} \longrightarrow \frac{7}{7}
\longrightarrow \frac{0}{7}$ indicate that the bow is to be drawn smoothly from
a point $\frac{6}{7}$ of the way up the bow up to the point of the bow
($\frac{7}{7}$) and then all the way back down to the talon ($\frac{0}{7}$).
This type of tablature allows for the specification of any point along the
compass of the bow and any type of travel of the bow between these points. The
BCPs used in the piece are all taken from values that divide the bow evenly
into four ($\frac{0}{4}, \frac{1}{4}, \frac{2}{4}, \frac{3}{4}, \frac{4}{4}$)
or seven ($\frac{0}{7}, \frac{1}{7}, \frac{2}{7}, \frac{3}{7}, \frac{4}{7},
\frac{5}{7}, \frac{6}{7}, \frac{7}{7}$) parts. The BCPs $\frac{0}{4}$ and
$\frac{0}{7}$ are synonyms for the talon; the BCPs $\frac{4}{4}$ and
$\frac{7}{7}$ are synonyms for the point. The four parametric windows in the
piece notated according to this tablature are all played half col legno tratto;
the choice of string contact point (tasto, ponticello and their variations) is
left to the preference of the players and should be determined once the sound
of the entire piece is familiar. The purpose of the notation is to allow for
dramatic changes of bow color: transitions and sudden changes from scratch to
extreme flautadando (and back again) are to be encouraged.

\textbf{Auxilliary instruments.} The players should be supplied with the
following auxilliary instruments:

\begin{tabu}{l l l}
\phantom{M} & clarinet: & small stone, slab of slate, large basket filled with lentils \\
            & violin: & small stone, slab of slate, eight cups filled with barley, bass drum (shared with viola) with brushes \\
            & viola: & small stone, slab of slate, bass drum (shared with violin) with brushes \\
            & cello: & small stone, slab of slate, large basket filled with dried beans \\
\end{tabu}

\textbf{Stonescratch.} At the end of the piece the cellist is asked to scratch
individuated marks across a piece of slate with a smooth stone. This moment is
marked ``stonescratch'' in the score. Follow the timings and loudnesses given
in the score.

\textbf{Stonecircle.} At different times in the piece all four players are
asked to move a smooth stone in cirlces across the surface of a piece of slate.
(Flagstone or the like can be used if slate is unavailable.) These moments are
marked ``stonecircle'' in the score. Each player should have their own stone
and their own piece of slate; the sound quality of the four pieces of slate
should differ slightly from each other. Circles drawn during performance should
be about as wide in diameter as the length of each player's hand. Different
rates of circling are given as fractions of $\pi$. Do not articulate the start
or stop of notes; the sound should be continuous and unaccented throughout:
what is important are loudnesses and differing rates of motion.

\textbf{Grainfall.} Eight times during the piece the violinist is requested to
pour the contents of a cup of barley to the floor. These moments are numbered
with Roman numerals and marked ``grainfall'' in the score. The action should be
performed so that the audience can see the motions that cause the resulting
sounds. The eight cups should match each other. Fill the cups before the
performance starts and line them up in a row. Each grainfall is a harbinger of
investigation: pour with mindfulness and return each empty cup to its place to
complete the action. The durations given in the score show when each grainfall
begins but the duration of each action should vary according to musical
context. A recepticle made of wood or clay should be positioned on stage near
the violinist's feet to catch the barley from each cup as it is poured. The
recepticle should be filled with a layer of barley about a knuckle deep before
the performance starts so that each pour results in the sound of grain falling
on grain; the recepticle can be placed on top of slab of styrofoam to amplify
the sound. Another grain may be substituted for barley so long as the grains
selected for the clarinetist, violinst and cellist all differ; see below for
the clarinet and cello.

\textbf{Graincircle.} At different times in the piece the cellist and the
clarinetist are asked to circle their hands in large baskets filled with grain.
These moments are marked ``graincircle'' in the score. The clarinetist's basket
should be filled with lentils; the cellist's basket should be filled with dried
beans; substitutions may be made so long as the grains selected for the
clarinetist, violinist and cellist all differ; see above for the violin. The
baskets should be made of fiber and be large enough to hold at least five or
ten kilograms of grain; baskets may be placed on top of slabs of styrofoam to
amplify the sound. Different rates of circling are given as fractions of $\pi$.
Do not articulate the start or stop of notes; the sound should be continuous
and unaccented throughout: what is important are loudnesses and differing rates
of motion.

\textbf{Accidentals.} Accidentals govern only one note. Natural signs are
inserted to clarify the spelling of different pitches that follow each other at
the same staff position.

\textbf{Transposition.} The bass clarinet sounds a major ninth lower than
written. The strings sound as written.

\end{textblock}

\begin{textblock}{23}(0, 23)

\textbf{Ikribu} was written for Distractfold, who gave the world premiere on 2
April 2016 in Paine Hall on the campus of Harvard University.

\end{textblock}

\end{document}
